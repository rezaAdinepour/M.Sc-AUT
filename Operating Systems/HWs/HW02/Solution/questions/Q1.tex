\section{سوال اول}




‫فرض‬ ‫کنید‬ ‫برنامه‬ ‫ای‬ ‫داریم‬ ‫که‬ ‫با‬ ‫چند‬ ‫نخ‬ ‫در‬ ‫حال‬ ‫اجرا‬ ‫می‬ ‫باشد‪،‬ ‫با‬ ‫توجه‬ ‫به‬ ‫این‬ ‫موضوع‬ ‫درستی‬ ‫و‬ ‫نادرستی‬ ‫عبارات‬ ‫زیر ‬‫را‬ ‫با‬ ‫ذکر‬ ‫دلیل‬ ‫مشخص‬ ‫کنید‬ ‫و‬ ‫توضیح‬ ‫دهید‬.‬‬
\begin{enumerate}
	\item ‫اگر‬ ‫یک‬ ‫نخ‬ ‫آرگومان‬ ‫های‬ ‫خاصی‬ ‫را‬ ‫به‬ ‫یک‬ ‫تابع‬ ‫در‬ ‫برنامه‬ ‫ارسال‬ ‫کند‪،‬‬ ‫این‬ ‫آرگومان‬ ‫ها‬ ‫برای‬ ‫نخ‬ ‫های‬ ‫دیگر‬ ‫قابل‬ مشاهده ‬‫هست‬
	
	\item ‫اگر‬ ‫یک‬ ‫نخ‬ ‫با‬ ‫استفاده‬ ‫از‬ ‫دستور‬ ‫‪\texttt{malloc}‬‬ ‫حافظه‬ ‫اضافی‬ ‫به‬ ‫خود‬ ‫تخصیص‬ ‫دهد‬ ‫می‬ ‫تواند ‬‫باعث‬ ‫به‬ ‫وجود‬ ‫آمدن‬ ‫خطای‬‫‪‬‬ \texttt{‫‪memory of out}‬‬ ‫برای‬ ‫نخ‬ ‫دیگری‬ ‫در‬ ‫برنامه‬ ‫شود‬


	\item ‫رشته‬ ‫های‬ ‫سطح‬ ‫کاربر‬ ‫توسط‬ ‫کتابخانه‬ ‫سطح‬ ‫کاربر‬ ‫برنامه‬ ‫ریزی‬‫ می‬شوند‬ ‫و‬ بدون‬ ‫اینکه‬ ‫هسته‬ ‫از‬ ‫عملیات‬ ‫آ‫ن ها‬ ‫مطلع‬ باشد ‬‫کار‬ ‫می‬‫کنند‬‬‬
	
	\item زمان \texttt{switch context} در رابطه با نخ‌های سطح هسته کمتر طول می‌کشد.
	
	
	\item نخ‌های سطح کاربر نمی‌توانند به صورت موازی واقعی \texttt{parallelism true} در سیستم‌های چند هسته‌ای اجرا شوند زیرا توسط یک رشته در سطح هسته مدیریت می‌شوند.
	
\end{enumerate}
\begin{qsolve}
	\begin{enumerate}
		\item درست\\توضیحات: در برنامه‌نویسی چندنخی، هر نخ دارای محیط اجرایی خود است و برخی متغیرها، مانند آرگومان‌های تابع، در حافظه اشتراکی میان نخ‌ها قرار می‌گیرند. بنابراین، اگر یک نخ آرگومانی را تغییر دهد، تغییرات ممکن است بر روی نخ‌های دیگر تأثیر بگذارد و آنها نیز مقدار تغییر یافته را مشاهده کنند
		
		
		\item نادرست\\ توضیحات: هر نخ دارای منطقه‌ای از حافظه برای استفاده خود است. اگر یک نخ بیش از حد حافظه را تخصیص دهد، حافظه موجود برای سایر نخ‌ها کاهش می‌یابد. این ممکن است منجر به خطای \texttt{memory of out} برای نخ‌های دیگری شود که نتوانند حافظه لازم را برای ادامه اجرا تخصیص دهند.
		
		
		\item درست\\ توضیحات: رشته‌های سطح کاربر توسط کتابخانه‌های سطح کاربر برنامه‌ریزی می‌شوند و اجرای آنها توسط هسته انجام نمی‌شود. این رشته‌ها به صورت مستقل و به طور همزمان اجرا می‌شوند و هسته از وجود و عملکرد آنها آگاهی ندارد
		
		
		\item نادرست\\ توضیحات:‌ زمان switch context بین نخ‌های سطح هسته Thread Level Kernel بیشتر از زمان switch context بین نخ‌های سطح کاربر است. زیرا در switch context بین نخ‌های سطح هسته، وضعیت کامل نخ، از جمله مجموعه ثبات نخ و محتوای ثبات شده، باید ذخیره شود و سپس وضعیت جدید نخ بارگذاری شود. این عملیات بیشترین زمان را می‌طلبد. در حالی که در switch context بین نخ‌های سطح کاربر،زمان کوتاهتری برای ذخیره و بارگذاری وضعیت نخ‌ها صرف می‌شود، زیرا این نخ‌ها توسط کتابخانه سطح کاربر برنامه‌ریزی مدیریت می‌شوند و اطلاعات کمتری برای تغییر وضعیت آنها نیاز است
		
	\end{enumerate}
\end{qsolve}

