\section{سوال دوم}

‫با‬ ‫در‬ ‫نظر‬ ‫گرفتن‬ ‫زمان 0 =‬ switch context در هر مورد، ‫نحوه‬ ‫زمان‌بندی‬ ‫پردازه‬ ‫ها‬ ‫را‬ ‫با‬ ‫یک‬ ‫نمودار‬
Gantt نشان دهید:

\begin{enumerate}
	\item ‫یک‬ ‫زمانبند‬ ‫چند‬ ‫لایه‬ ‫ای‬ ‫داریم‬ ‫که‬ ‫در‬ ‫آن‬ ‫پردازه‬ ‫ها‬ ‫دوبار‬ ‫وقت‬ ‫دارند‬ ‫در‬ ‫لایه‬ ‫اول‬ ‫زمان‌بندی‬ ‫شوند‪ ،‬اگر‬ ‫کار‬ ‫آنها‬ ‫به پایان نرسید، ‫وارد‬ ‫الیه‬ ‫دوم‬ ‫شده‬ ‫و‬ ‫یک‬ ‫بار‬ ‫هم‬ ‫وقت‬ ‫دارند‬ ‫آنجا‬ ‫زمان‌بندی‬ ‫شوند‬ ‫و‬ ‫در‬ ‫نهایت‬ ‫اگر‬ ‫کار‬ ‫آن‌ها‬ ‫همچنان‬ ‫به‬ ‫پایان‬‬ ‫نرسید‬‫ وارد‬ ‫لایه‬ ‫آخر‬ ‫شده‬ ‫و‬ ‫تا‬ ‫زمانی‬ ‫که‬ ‫کار‬ ‫پردازشی‬ ‫آن‌ها‬ ‫به‬ ‫اتمام‬ ‫برسد‬ ‫در‬ ‫آنجا‬ ‫زمانبندی‬ ‫می‬ ‫شوند‪،‬داریم‪:‬‬
	
	لایه اول: RR با کوانتوم زمانی ۴\\
	لایه دوم: ‌RR با کوانتوم زمانی ۸\\
	لایه سوم: FCFS\\
	و پردازه ها مطابق جدول زیر باشند:

\begin{latin}
	\begin{center}
		\begin{tabular}{||c| c c||} 
			\hline
			- & Arrival time & ‫‪CPU burst \\ [0.5ex] 
			\hline\hline
			P0 & 0 & 13 \\ 
			\hline
			P1 & 0 & 10 \\ 
			\hline
			P2 & 8 & 17 \\
			\hline
			P3 & 24 & 40 \\
			\hline
			P4 & 40 & 16 \\
			\hline
			P5 & 80 & 4 \\ [1ex] 
			\hline
		\end{tabular}
	\end{center}
\end{latin}

‫و‬‫ همه‬ ‫جا‬ ‫اولویت‬ ‫تخصیص‬ ‫پردازنده‬ ‫با‬ ‫پردازه‬ ‫ای‬ ‫باشد‬ ‫که‬ ‫کار‬ ‫در‬ ‫لایه‬ ‫بالاتر‬ ‫دارد‬ ‫و‬ ‫زمانبند‬ ‫غیرپسگیر باشد.


	\item یک زمان‌بندی کاملا پسگیر داریم که ‫بر‬ ‫اساس‬ ‫اولویت‬ ‫کار‬ ‫می‬ ‫کند‬ ‫و‬ ‫اگر‬ ‫دو‬ ‫پردازه‬ ‫اولویت‬ ‫یکسان‬ ‫داشته‬‫باشند‬ ‫بر‬ ‫اساس‬ ‫‪SRT‬‬ ‫عمل‬ ‫می‬ ‫کند‪.‬‬ ‫در‬ ‫این‬ ‫زمانبند‪،‬‬ ‫به‬ ‫ازای‬ ‫هر‬ ‫‪20‬‬ ‫واحد‬ ‫زمانی‪،‬در‬ ‫صورتی‬ ‫که‬ ‫کار‬ ‫پردازه‬ ‫به‬ ‫اتمام‬‫نرسیده‬ ‫باشد‪،‬‬ ‫یک‬ ‫واحد‬ ‫از‬ ‫عدد‬ ‫اولویت‬ ‫پردازه‬ ‫ها‬ ‫کم‬ ‫شود‬ ‫و‬ ‫پردازه‬ ‫ها‬ ‫مطابق‬ ‫جدول‬ ‫زیر‬ ‫هستند‬‬
	
	\begin{latin}
		\begin{center}
			\begin{tabular}{||c| c c c||} 
				\hline
				- & Arrival time & ‫‪CPU burst & ‫‪Priority‬‬ ‫‪number‬‬ \\ [0.5ex] 
				\hline\hline
				P0 & 45 & 9 & 2 \\ 
				\hline
				P1 & 0 & 15 & 4 \\ 
				\hline
				P2 & 10 & 30 & 3 \\
				\hline
				P3 & 15 & 20 & 3 \\
				\hline
				P4 & 50 & 26 & 4 \\ [1ex] 
				\hline
			\end{tabular}
		\end{center}
	\end{latin}
\end{enumerate}


\begin{qsolve}
	نمودار Gantt برای لایه‌های چندگانه با الگوریتم‌های RR و :FCFS
	\begin{enumerate}
		\item محاسبه زمان اجرای هر پردازه در هر لایه:
		\begin{itemize}
			\item لایه اول: \\
			\begin{latin}
				P0: 13, P1: 10, P2: 4 
			\end{latin} (زمان اجرا به اتمام می‌رسد)
			
			\item لایه دوم: \\
			\begin{latin}
			P2: 13, P3: 8, P4: 16
			\end{latin} (زمان اجرا به اتمام می‌رسد)
			
			\item لایه سوم: \\
			\begin{latin}
			P3: 17, P5: 4
			\end{latin} (زمان اجرا به اتمام می‌رسد)
		\end{itemize}
		
		
		\item نمودار :Gantt
		\begin{latin}
			layer 1: $| P0 | P1 | P2 | P2 | P3 | P4 |$ \\
			layer 2: $|\ \ \ \ \ P2 \ \ \ \ |\ \ P3 \ \ | P4 |$ \\ 
			layer 3: $|\ \ \ \ \ \ \ P3 \ \ \ \ \ \ |\ \ P5 \ \ |$ \\
			Time:  $ 0 \ \ 4 \ \ 8 \ \ 12 \ \ 20 \ \ 37 \ \ 53 \ \ 57 \ \ 61 $
		\end{latin}
	\end{enumerate}
	
	
	نمودار Gantt برای زمانبندی کاملاً پسگیر با اولویت‌های SRT و Priority:
	\begin{enumerate}
		\item محاسبه زمان اجرای هر پردازه در هر لحظه:
		\begin{itemize}
			\item زمان ۰: \\
			\begin{latin}
				P1: 15, P2: 10
			\end{latin}
			
			\item زمان ۱۰: \\
			\begin{latin}
				P2: 20, P3: 20
			\end{latin}
			
			\item زمان ۲۰: \\
			\begin{latin}
				P0: 9, P3: 10
			\end{latin}
			
			\item زمان ۳۰: \\
			\begin{latin}
				P3: 10, P4: 26	
			\end{latin}
			
			\item زمان ۵۶: \\
			\begin{latin}
				P4: 4
			\end{latin}
		\end{itemize}
	\end{enumerate}
\end{qsolve}

\begin{qsolve}
	\begin{enumerate}
		\item نمودار :Gantt \\
		\begin{latin}
			Time: $ 0 \ \ 10 \ \ 20 \ \ 30 \ \ 40 \ \ 50 \ \ 50 $ \\
			layer 1: $ | P1 \ | P2 \ | P3 \ | P4 \ | P0 \ | P3 \ | P4 | $	
		\end{latin}
	\end{enumerate}
\end{qsolve}