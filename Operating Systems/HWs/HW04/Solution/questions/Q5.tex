\section{سوال پنجم}

\begin{enumerate}
	\item با توجه به ۵ پارتیشن حافظه ۱۰۰ کیلوبایتی، ۵۰۰ کیلوبایتی، ۲۰۰ کیلوبایتی، ۳۰۰ کیلوبایتی و ۶۰۰ کیلوبایتی (به ترتیب) هریک از الگوریتم‌های bestFit و firstFit و worstFit به ترتیب پردازه‌های ۱۱۲، ۴۱۷، ۲۱۲ و ۴۲۶ کیلوبایتی را چگونه قرار می‌دهند و بگویید کدام یک از الگوریتم‌های ذکر شده بهینه‌تر است؟
	
	
	
	\item با توجه به جدول زیر، آدرس فیزیکی مربوط به آدرس‌های منطقی زیر را به‌دست آورید.
	
	\begin{latin}
		\begin{enumerate}
			\item 0, 430
			\item 1, 10
			\item 2, 500
			\item 3, 400
			\item 4, 112
		\end{enumerate}
		\begin{center}
			\begin{tabular}{||c c c||} 
				\hline
				Segment & Base & Length \\ [0.5ex] 
				\hline\hline
				0 & 219 & 600\\ 
				
				1 & 2300 & 14 \\
				
				2 & 90 & 100 \\
				
				3 & 1327 & 580 \\
				
				4 & 1954 & 96 \\ [1ex] 
				\hline
			\end{tabular}
		\end{center}
	\end{latin}
	
\end{enumerate}







\begin{qsolve}
	
	\begin{enumerate}
		\item برای قرار دادن پردازه‌ها با اندازه‌های ۱۱۲، ۴۱۷، ۲۱۲ و ۴۲۶ کیلوبایت در الگوریتم‌های bestFit و firstFit و ،worstFit می‌توان به صورت زیر عمل کرد:
		
		
		\begin{enumerate}
			\item \textbf{پردازه ۱۱۲ کیلوبایتی:}
			\begin{enumerate}
				\item \textbf{:bestFit} بلوک ۱ با بازه بین ۲۰۰ تا ۲۹۹ کیلوبایت
				\item \textbf{:firstFit} بلوک ۱ با بازه بین ۲۰۰ تا ۲۹۹ کیلوبایت
				\item \textbf{:worstFit} بلوک ۳ با بازه بین ۴۰۰ تا ۵۹۹ کیلوبایت
			\end{enumerate}
			
			\item \textbf{پردازه ۴۱۷ کیلوبایتی:}
			\begin{enumerate}
				\item \textbf{:bestFit} بلوک ۴ با بازه بین ۱۱۲ تا ۲۲۷ کیلوبایت
				\item \textbf{:firstFit} بلوک ۱ با بازه بین ۲۰۰ تا ۵۹۹ کیلوبایت
				\item \textbf{:worstFit} بلوک ۳ با بازه بین ۴۰۰ تا ۵۹۹ کیلوبایت
			\end{enumerate}
			
			\item \textbf{پردازه ۲۱۲ کیلوبایتی:}
			\begin{enumerate}
				\item \textbf{:bestFit} بلوک ۱ با بازه بین ۲۰۰ تا ۲۹۹ کیلوبایت
				\item \textbf{:firstFit} بلوک ۱ با بازه بین ۲۰۰ تا ۲۹۹ کیلوبایت
				\item \textbf{:worstFit} بلوک ۳ با بازه بین ۴۰۰ تا ۵۹۹ کیلوبایت
			\end{enumerate}
			
			\item \textbf{پردازه ۴۲۶ کیلوبایتی:}
			\begin{enumerate}
				\item \textbf{:bestFit} بلوک ۵ با بازه بین ۹۶ تا ۱۹۱ کیلوبایت
				\item \textbf{:firstFit} بلوک ۱ با بازه بین ۲۰۰ تا ۵۹۹ کیلوبایت
				\item \textbf{:worstFit} بلوک ۳ با بازه بین ۴۰۰ تا ۵۹۹ کیلوبایت
			\end{enumerate}
		\end{enumerate}
	\end{enumerate}	
\end{qsolve}






\begin{qsolve}
	
	\begin{enumerate}
		\item برای به‌دست آوردن آدرس فیزیکی مربوط به آدرس‌های منطقی زیر، می‌توانیم از جدول Segment، Base و Length استفاده کنیم:
		
			\begin{enumerate}
				\item \textbf{آدرس منطقی (0, 430):}
				 آدرس فیزیکی = Base + Offset\\
				  آدرس فیزیکی = 219 + 430 = 649
			
			\item \textbf{آدرس منطقی (1, 10):}
			 آدرس فیزیکی = 2300 + 10 = 2310
			 
			 \item \textbf{آدرس منطقی (2, 500):}
			  آدرس فیزیکی = 90 + 500 = 590
			  
			  \item \textbf{آدرس منطقی (3, 400):}
			   آدرس فیزیکی = 1327 + 400 = 1727
			   
			   \item \textbf{آدرس منطقی (4, 112):}
				 آدرس فیزیکی = 1954 + 112 = 2066
			\end{enumerate}
	\end{enumerate}
\end{qsolve}


