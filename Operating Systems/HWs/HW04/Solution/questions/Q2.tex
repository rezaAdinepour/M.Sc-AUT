\section{سوال دوم}

چه تعداد خروجی ۱۰ کاراکتره برای دو پردازه زیر که موازی با یکدیگر اجرا می‌شود ممکن است؟ کلیت الگو آن را ذکر کنید. (مقدار اولیه هر دو سمافور ۰ و غیر باینری هستند)


\begin{latin}
\begin{lstlisting}
P1:                    P2:
while(true)            while(true)
{                      {
  signal(S1);            signal(S2);
  print("A");            print("B");
  wait(S2);              wait(S1);
  print("A");            print("B");
}	                   }

\end{lstlisting}
\end{latin}

\textbf{راهنمایی:‌ }از اصل جمع و حالت بندی بر اساس وضعیت ۴ کاراکتر‌های چاپ شده استفاده کنید.

\begin{qsolve}
برای تعیین تعداد خروجی‌های ۱۰ کاراکتره برای دو پردازه P1 و P2، با استفاده از اصل جمع و حالت بندی بر اساس وضعیت چهار کاراکتر چاپ شده، به طور مشابه با پرسش قبل عمل می‌کنیم. در اینجا نیز، دو کاراکتر "A" و "B" را در نظر می‌گیریم.

حالت‌های ممکن برای چهار کاراکتر چاپ شده عبارتند از:
\begin{enumerate}
	\item :AABB P1 یک بار چاپ "A" و سپس P2 یک بار چاپ "B"
	\item :ABAB P1 یک بار چاپ "A" و سپس P2 یک بار چاپ "B"
	\item :BAAB P2 یک بار چاپ "B" و سپس P1 یک بار چاپ "A"
	\item :BBAA P2 یک بار چاپ "B" و سپس P1 یک بار چاپ "A"
\end{enumerate}

با توجه به اصل جمع و حالت بندی، تعداد کل خروجی‌های ۱۰ کاراکتره برابر است با تعداد حالت‌های ممکن برای چهار کاراکتر چاپ شده، که در اینجا برابر با 4 است.

بنابراین، تعداد خروجی‌های ۱۰ کاراکتره برای دو پردازه P1 و P2، برابر با 4 است و الگوی کلی آن عبارت است از: ,AABB ,ABAB ,BAAB BBAA
.

\end{qsolve}