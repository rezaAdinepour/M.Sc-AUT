%           ******************************************************
%          **   course         : Advanced Computer Architecture  **
%         ***   HomeWork       : 01                              ***
%        ****   Topic          : Introduction to Gem5            ****
%        ****   AUTHOR         : Reza Adinepour                  ****
%         ***   Student ID:    : 402131055                       ***
%          **   Github         : github.com/rezaAdinepour/       **
%           ******************************************************

\documentclass[12pt]{exam}

\usepackage{setspace}
\usepackage{listings}
\usepackage{graphicx,subfigure,wrapfig}
\usepackage{multirow}
\usepackage{blindtext}
\usepackage{multicol}
\setlength{\columnsep}{1cm}
\usepackage{hyperref}
\hypersetup{
	colorlinks=true,
	linkcolor=blue,
	filecolor=magenta,      
	urlcolor=cyan,
	pdftitle={Overleaf Example},
	pdfpagemode=FullScreen,
}
\usepackage{xcolor}
\usepackage{code-style}
\usepackage{caption}


\DeclareCaptionFont{white}{\color{white}}
\DeclareCaptionFormat{listing}{%
	\parbox{\textwidth}{\colorbox{gray}{\parbox{\textwidth}{#1#2#3}}\vskip-4pt}}
\captionsetup[lstlisting]{format=listing,labelfont=white,textfont=white}
\lstset{frame=lrb,xleftmargin=\fboxsep,xrightmargin=-\fboxsep}


\usepackage[margin=20mm]{geometry}
\usepackage{xepersian}
\settextfont{XB Niloofar}

\newcommand{\class}{درس معماری کامپیوتر پیشرفته}
\newcommand{\term}{نیم‌سال اول ۰۲-۰۳}
\newcommand{\college}{دانشکده مهندسی کامپیوتر}
\newcommand{\prof}{استاد: دکتر فربه}

\singlespacing
\parindent 0ex

\lstset{
keywordstyle=\textbf,
identifierstyle=, 
stringstyle=\ttfamily,
commentstyle=\color{LimeGreen}, 
stringstyle=\ttfamily,
numberstyle=\footnotesize,
showstringspaces=false} 
\begin{document}


% -------------------------------------------------------
%  Thesis Information
% -------------------------------------------------------

\newcommand{\ThesisType}
{سمینار}  % پایان‌نامه / رساله
\newcommand{\ThesisDegree}
{کارشناسی ارشد گرایش معماری کامپیوتر}  % کارشناسی / کارشناسی ارشد / دکتری
\newcommand{\ThesisMajor}
{مهندسی کامپیوتر}  % مهندسی کامپیوتر
\newcommand{\ThesisTitle}
{تمرین سری ۱: مقدمه‌ای بر GEM5}
\newcommand{\ThesisAuthor}
{رضا آدینه پور - 402131055}
\newcommand{\ThesisSupervisor}
{جناب آقای دکتر فربه}
\newcommand{\ThesisDate}
{\today}
\newcommand{\ThesisDepartment}
{دانشکده مهندسی کامپیوتر}
\newcommand{\ThesisUniversity}
{دانشگاه صنعتی امیرکبیر}

% -------------------------------------------------------
%  English Information
% -------------------------------------------------------

\newcommand{\EnglishThesisTitle}{A Standard Template for Course Exercise}


\pagestyle{empty}

\begin{center}

\includegraphics[scale=0.2]{images/logo.png}

%\vspace{0.5cm}
%\ThesisUniversity \\[-0.3em]
\vspace{0.3cm}
\large\ThesisDepartment\\

\begin{large}
\vspace{0.5cm}


%\ThesisMajor

\end{large}

\vspace{1.5cm}

{عنوان:}\\[1.2em]
{\LARGE\textbf{\ThesisTitle}}\\ 
\vspace{1cm}
%\begin{latin}
%{\Large\textbf\EnglishThesisTitle}
%\end{latin}

\vspace{2cm}

{نگارش}\\[.5em]
{\large\textbf{\ThesisAuthor}}

\vspace{1.5cm}

{استاد راهنما}\\[.5em]
{\large\textbf{\ThesisSupervisor}}

\vspace{1cm}



\vspace{2cm}

\ThesisDate

\end{center}

\newpage


% These commands set up the running header on the top of the exam pages
\pagestyle{head}
\firstpageheader{}{}{}
\runningheader{صفحه \thepage\ از \numpages}{}{\class}
\runningheadrule
\vspace{0pt}

• پرسش‌های خود را می‌توانید در  تالار ایجاد شده در سایت درس مطرح کنید.\\
•  نوشتن نام خود را فراموش نفرمایید.\\


%\baselineskip = 9mm

\begin{questions}
	\pointpoints
	
	
	\question
	کار برنامه زیر چیست؟ 
\end{questions}

\begin{multicols}{1}
	[
	\section{بخش اول}
	در اینجا می‌توان مقدمه ای در مورد کل پاراگراف مورد نظر نوشت.
	]
	این متن جهت تست نوشته شده است این متن جهت تست نوشته شده است این متن جهت تست نوشته شده است این متن جهت تست نوشته شده است این متن جهت تست نوشته شده است این متن جهت تست نوشته شده است این متن جهت تست نوشته شده است این متن جهت تست نوشته شده است این متن جهت تست نوشته شده است این متن جهت تست نوشته شده است این متن جهت تست نوشته شده است این متن جهت تست نوشته شده است این متن جهت تست نوشته شده است این متن جهت تست نوشته شده است این متن جهت تست نوشته شده است این متن جهت تست نوشته شده است این متن جهت تست نوشته شده است این متن جهت تست نوشته شده است این متن جهت تست نوشته شده است این متن جهت تست نوشته شده است این متن جهت تست نوشته شده است این متن جهت تست نوشته شده است این متن جهت تست نوشته شده است این متن جهت تست نوشته شده است این متن جهت تست نوشته شده است این متن جهت تست نوشته شده است این متن جهت تست نوشته شده است این متن جهت تست نوشته شده است این متن جهت تست نوشته شده است این متن جهت تست نوشته شده است این متن جهت تست نوشته شده است این متن جهت تست نوشته شده است این متن جهت تست نوشته شده است این متن جهت تست نوشته شده است این متن جهت تست نوشته شده است این متن جهت تست نوشته شده است این متن جهت تست نوشته شده است این متن جهت تست نوشته شده است این متن جهت تست نوشته شده است این متن جهت تست نوشته شده است این متن جهت تست نوشته شده است این متن جهت تست نوشته شده است این متن جهت تست نوشته شده است این متن جهت تست نوشته شده است این متن جهت تست نوشته شده است این متن جهت تست نوشته شده است این متن جهت تست نوشته شده است این متن جهت تست نوشته شده است این متن جهت تست نوشته شده است این متن جهت تست نوشته شده است این متن جهت تست نوشته شده است این متن جهت تست نوشته شده است این متن جهت تست نوشته شده است 
\end{multicols}


\begin{latin}
	
	\lstinputlisting[
	caption=Example Python,
	label={lst:listing-py},
	language=python,
	style=myStyle,
	]{codes/code2.py}

\end{latin}



\end{document}