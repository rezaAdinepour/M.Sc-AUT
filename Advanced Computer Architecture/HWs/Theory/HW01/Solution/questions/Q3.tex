\section{سوال سوم}

تلفن همراه وظایف متفاوتی از جمله پخش موسیقی،‌ پخش ویدئو و خواندن ایمیل را انجام می‌دهد. این وظایف،‌ محاسبات زیادی نیاز دارد. عمر باتری و گرمای بیش از حد دو مشکل رایج برای تلفن همراه است. بنابر این کاهش مصرف برق و انرژی برای تلفن‌های همراه بسیار مهم است. در این مسئله ما درنظر می‌گیریم که وقتی کاربر از تلفن با ظرفیت کامل محاسباتی خود استفاده نمی‌کند، چه کاری انجام دهیم. برای این مشکلات سناریوی غیر واقعی را بررسی خواهیم کرد که تلفن همراه واحد پردازش تخصصی ندارد در عوض، دارای یک واحد پردازش چهار هسته ای خاص منظوره است. هر هسته در استفاده کامل ۰٫۵ وات استفاده می‌کند. برای کار‌های مربوط به ایمیل، پردازنده Core ،Quad ۸ برابر سریع‌تر است.

\begin{enumerate}
	\item چه مقدار انرژی و توان دینامیکی در مقایسه با کار با تمام توان نیاز است؟ ابتدا فرض کنید که Core ،Quad ۱٫۸ زمان کار می‌کند و در زمان‌های دیگر بیکار است. یعنی زمان در ۷٫۸ مواقع غیرفعال است و در آن زمان هیچ نشتی ای رخ نمی‌دهد. انرژی دینامیکی کل و همچنین توان دینامیکی را حین کار کردن هسته مقایسه کنید.
	
	\begin{qsolve}
		انرژی برابر است با $\frac{1}{8}$ توان که در این مثال بدون تغییر باقی می‌ماند
	\end{qsolve}
	
	
	
	
	
	
	
	\item با استفاده از مقیاس فرکانس و ولتاژ چه انرژی و توان دینامیکی مورد نیاز است؟ فرض کنید فرکانس و ولتاژ هردو در کل زمان به ۱٫۸ کاهش می‌یابند.
	
	\begin{qsolve}
		\begin{equation}
			Energy: \frac{Energy_{new}}{Energy_{old}} = \frac{(\frac{1}{8} \times V)^2}{V^2} = \frac{1}{64} = \mathcolorbox{yellow}{0.015625}
		\end{equation}
		
		\begin{equation}
			Power: \frac{Power_{new}}{Power_{old}} = 0.156 \times \frac{(\frac{1}{8} \times f)}{f} = \frac{0.015625}{8} = \mathcolorbox{yellow}{0.0195}
		\end{equation}
	\end{qsolve}
	
	
	
	
	
	
	
	
	\item حال فرض کنید ولتاژ ممکن است کمتر از ۵۰ درصد ولتاژ اولیه کاهش نیابد. این ولتاژ به عنوان کف ولتاژ نامیده می‌شود و هر ولتاژ کمتر از آن را از دست می‌دهد. بنابراین، درحالی که فرکانس می‌تواند مدام کاهش یابد، ولتاژ نمی‌تواند. Saving Power و انرژی دینامیکی را محاسبه کنید؟
	
	\begin{qsolve}
		\begin{equation}
			Energy: \frac{Energy_{new}}{Energy_{old}} = \frac{(\frac{1}{2} \times 	V)^2}{V^2} = \frac{1}{4} = \mathcolorbox{yellow}{0.25}
		\end{equation}
		
		\begin{equation}
			Power: \frac{Power_{new}}{Power_{old}} = 0.25 \times \frac{(\frac{1}{8} \times 	f)}{f} = \frac{0.25}{8} = \mathcolorbox{yellow}{0.03125}
		\end{equation}
	\end{qsolve}
	
	
	
	
	
	
	
	
	
	
	\item چه مقدار انرژی با رویکرد Silicon Dark مصرف می‌شود؟ این شامل ایجاد سخت افزار تخصصی ASIC برای هر کار و راه‌اندازی توان آن عناصر در زمانی است که استفاده نمی‌شوند. فقط یک هسته خاص منظوره ارائه می‌شود و بقیه تراشه با واحد‌های تخصصی پر می‌شود. برای ایمیل، یک هسته برای ۲۵٪
	 زمان کار می‌کند و برای ۷۵٪ دیگر به‌طور کامل با Gating Power خاموش می‌شود. در ۷۵٪دیگر مواقع، یک واحد تخصصی ASIC که به ۲۰ درصد انرژی یک هسته نیاز دارد، کار می‌کند.
	
	\begin{qsolve}
		برای تک هسته ۰٫۲۵ توان اصلی مصرف که در ۰٫۲۵ زمان اجرا می‌شود. بنابر این داریم: 
		\begin{equation}
			0.25 \times 0.25 + (0.25 \times 0.2) \times 0.75 = 0.0625 + 0.0375 = \mathcolorbox{yellow}{0.1}
		\end{equation}
	\end{qsolve}
\end{enumerate}